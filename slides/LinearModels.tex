\documentclass{beamer}

\usepackage{lads}
\setbeamertemplate{navigation symbols}{}

\title{Linear Models}
\date{}
\author{BIOE 210}

\begin{document}

\maketitle

\begin{frame}
\frametitle{Is there a better method?}

We can derive estimates for any type of linear model. But the amount of math scales cubicly with the number of parameters (why?).

\bigskip
Instead, let's turn to a matrix formalism for linear models.	
\end{frame}

\begin{frame}
\frametitle{A matrix formalism for linear models}
Let's write out one equation for each observation of the model $y = \beta_0 + \beta_1x$.
\begin{align*}
	-0.05 = \beta_0 + 0.07\beta_1 + \epsilon_1 \\
	0.40 = \beta_0 + 0.16\beta_1 + \epsilon_2 \\
	0.66 = \beta_0 + 0.48\beta_1 + \epsilon_3 \\
	0.65 = \beta_0 + 0.68\beta_1 + \epsilon_4 \\
	1.12 = \beta_0 + 0.83\beta_1 + \epsilon_5
\end{align*}

\pause
\[ \begin{pmatrix} -0.05\\ \phan0.40\\ \phan0.66\\ \phan0.65\\ \phan1.12 \end{pmatrix} = \begin{pmatrix} 1&0.07\\ 1&0.16\\ 1&0.48\\ 1&0.68\\ 1&0.83 \end{pmatrix} \vectwo{\beta_0}{\beta_1} + \begin{pmatrix} \epsilon_1\\ \epsilon_2\\ \epsilon_3\\ \epsilon_4\\ \epsilon_5 \end{pmatrix} \]
\pause
\[ \Vy = \V{X}\Vbeta+\Vepsilon \]	
\end{frame}

\begin{frame}
\frametitle{Solving the linear system}

A few points about $\Vy = \V{X}\Vbeta+\Vepsilon$:
\begin{itemize}
	\item The unknowns are \Vbeta, not \V{X}.
	\item The coefficient matrix \V{X}\ is called the \emph{model matrix}.
	\item The design matrix \V{X}\ is rarely square.
\end{itemize}

\pause
The solution to this system that minimizes the errors in \Vepsilon\ is 
\[ \Vbeta = \VX^+\Vy \]
where $\VX^+$ is the \emph{pseudoinverse} of \VX.
\end{frame}

\begin{frame}
\frametitle{For next time}
\begin{itemize}
	\item We will see how to calculate the pseudoinverse in Part III.
	\item Next time we will demonstrate how to formulate and solve more complex linear models.
\end{itemize}	
\end{frame}


\end{document}
