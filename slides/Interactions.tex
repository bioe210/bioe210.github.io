\documentclass{beamer}

\usepackage{lads}
\setbeamertemplate{navigation symbols}{}

\title{Interactions in Linear Models}
\date{BIOE 210}

\begin{document}

\maketitle

\begin{frame}
\frametitle{What is an interaction?}	

Imagine we're modeling the response ($y$) from two input variables, $x_1$ and $x_2$. The simplest model is

\[ y = \beta_1x_1 + \beta_2x_2 + \epsilon \]

\pause
The coefficient $\beta_1$ measures the effect of $x_1$ and $\beta_2$ measures the effect of $x_2$. These effects are \textbf{independent}.

\bigskip
\pause
What is there is another effect that depends on both $x_1$ and $x_2$? This is an \textbf{interaction} between $x_1$ and $x_2$.
\end{frame}

\begin{frame}
\frametitle{How do we model interactions?}

We model the interaction of $x_1$ and $x_2$ using the product of these variables.
\[ y = \beta_1x_1 + \beta_2x_2 + \beta_{12}x_1x_2 + \epsilon \]

The coefficient $\beta_{12}$ is the effect size of the interaction.

\pause
\bigskip
Why do we multiply $x_1$ and $x_2$? There are at least two ways to interpret this term.
\end{frame}

\begin{frame}
\frametitle{The coded factor interpretation}

Often we set up design matrices using \textbf{coded variables}. If we're testing the variable at two levels, we code the variable as ``on/off"~($\{0,1\}$) or ``low/high"~($\{-1,+1\}$).

\pause
\bigskip
\begin{columns}
\begin{column}{0.55\textwidth}
on/off $\rightarrow$ interaction when both ``on"
\begin{center}
\begin{tabular}{cc|c}
	$x_1$ & $x_2$ & $x_1x_2$ \\
	\hline
	0 & 0 & 0 \\
	0 & 1 & 0 \\
	1 & 0 & 0 \\
	1 & 1 & 1
\end{tabular}
\end{center}
\end{column}

\pause
\begin{column}{0.5\textwidth}
high/low $\rightarrow$ interaction when both ``high" or both ``low"
\begin{center}
\begin{tabular}{cc|c}
	$x_1$ & $x_2$ & $x_1x_2$ \\
	\hline
	$-1$ & $-1$ & $+1$ \\
	$-1$ & $+1$ & $-1$ \\
	$+1$ & $-1$ & $-1$ \\
	$+1$ & $+1$ & $+1$
\end{tabular}
\end{center}
\end{column}
\end{columns}

\end{frame}

\begin{frame}
\frametitle{The augmented slope interpretation}

We can also interpret the interaction as one variable changing the effect of the other variable.

\begin{align*}
	y &= \beta_1x_1 + \beta_2(x_1)x_2 + \epsilon \\
	 &= \beta_1x_1 + (\beta_2 + \beta_{12}x_1)x_2 + \epsilon \\
	 &= \beta_1x_1 + \beta_2x_2 + \beta_{12}x_1x_2 + \epsilon
\end{align*}
\end{frame}

\begin{frame}
\frametitle{Things to remember about interactions}

\begin{itemize}
	\item Interaction are modeled as the product of variables.
	\item The interaction effect is ``above and beyond" the independent effects (synergy/super-additivity, antagonism/sub-additivity).
	\item Higher-order interactions are possible (e.g. $x_1x_2x_3$), but these are rare.
\end{itemize}
\end{frame}

\end{document}
