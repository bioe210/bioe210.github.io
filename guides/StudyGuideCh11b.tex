\documentclass[12pt,ragged]{pajarticle}

%\usepackage{showframe}

\usepackage{lads}

\setlength{\parindent}{0em}
\setlength{\parskip}{1em}

\begin{document}

\AtEndDocument{\section*{Solutions}}

\section*{\hbox{Chapter 11: Vector Spaces, Span, and Basis (Part~II)}}

\begin{question}{1}{%
Verify that the vectors
\[ \Vv_1 = \vecthree{-1}{1}{1}, \Vv_2 = \vecthree{1}{-1}{2}, \Vv_3 = \vecthree{1}{1}{0} \]
are mutually orthogonal.}{%
The vectors are mutually orthogonal if every vector is orthogonal to every other vector. We can check orthogonality by computing dot products.
\begin{align*}
	\Vv_1\cdot\Vv_2 &= (-1)(1) + (1)(-1) + (1)(2) = 0 \\
	\Vv_1\cdot\Vv_3 &= (-1)(1) + (1)(1) + (1)(0) = 0 \\
	\Vv_2\cdot\Vv_3 &= (1)(1) + (-1)(1) + (2)(0) = 0	
\end{align*}
}
\end{question}

\begin{question}{2}{%
Are the above vectors orthonormal?}{%
We know the vectors are mutually orthogonal. For the vectors to be an orthonormal set, every vector needs to be a unit vector.
\begin{align*}
	\norm{\Vv_1} &= \sqrt{(-1)^2 + 1^2 + 1^2} = \sqrt{3} \ne 1 \\
	\norm{\Vv_2} &= \sqrt{1^2 + (-1)^2 + (2)^2} = \sqrt{6} \ne 1 \\
	\norm{\Vv_3} &= \sqrt{1^2 + 1^2 + 0^2} = \sqrt{2} \ne 1
\end{align*}
None of the vectors are unit vectors. We knew the vectors were not an orthonormal set after computing the first magnitude.
}
\end{question}

\begin{question}{3}{%
Make the vectors $\Vv_1$, $\Vv_2$, and $\Vv_3$ into	an orthonormal set.}{%
We showed that the three vectors are mutually orthogonal. All that remains is to normalize each vector by dividing by its magnitude.
\begin{align*}
	\hat{\Vv}_1 &= (1/\sqrt{3})\Vv_1 \\
	\hat{\Vv}_2 &= (1/\sqrt{6})\Vv_2 \\
	\hat{\Vv}_3 &= (1/\sqrt{2})\Vv_3	
\end{align*}

}
\end{question}

\begin{question}{4}{%
Are the vectors $\hat{\Vv}_1$, $\hat{\Vv}_2$, and $\hat{\Vv}_3$ a basis for $\Reals^3$? How about the vectors $\Vv_1$, $\Vv_2$, and $\Vv_3$?}{
A basis in $\Reals^3$ requires three vectors. The vectors $\hat{\Vv}_1$, $\hat{\Vv}_2$, and $\hat{\Vv}_3$ are mutually orthogonal, so they are also linearly independent (see the beginning of \S~11.5). Any three linearly independent vectors form a basis in $\Reals^3$. This argument requires only that the vectors be mutually orthogonal, not orthonormal, so the vectors $\Vv_1$, $\Vv_2$, and $\Vv_3$ also form a basis in $\Reals^3$.
}
\end{question}


\begin{question}{5}{%
Decompose the vector
\[ \Vu = \vecthree{3}{4}{-2} \]
onto the basis vectors $\hat{\Vv}_1$, $\hat{\Vv}_2$, and $\hat{\Vv}_3$.}{
We are looking for coefficients $a_1$, $a_2$, and $a_3$ such that
\[ \Vu = a_1\hat{\Vv}_1 + a_2\hat{\Vv}_2 + a_3\hat{\Vv}_3 \]
Our basis is orthonormal so we can apply the shortcut formula using dot products.
\begin{align*}
	a_1 &= \Vu\cdot\hat{\Vv}_1 \\
		&= \vecthree{3}{4}{-2}\cdot \frac{1}{\sqrt{3}}\vecthree{-1}{1}{1} \\
		&= (-3 + 4 - 2)/\sqrt{3} = -1/\sqrt{3} \\
	a_2 &= \Vu\cdot\hat{\Vv}_2 \\
		&= \vecthree{3}{4}{-2}\cdot \frac{1}{\sqrt{6}}\vecthree{1}{-1}{2} \\
		&= (3 - 4 - 4)/\sqrt{6} = -5/\sqrt{6} \\
	a_3 &= \Vu\cdot\hat{\Vv}_3 \\
		&= \vecthree{3}{4}{-2}\cdot \frac{1}{\sqrt{2}}\vecthree{1}{1}{0} \\
		&= (3 + 4 + 0)/\sqrt{2} = 7/\sqrt{2} \\
\end{align*}
We can confirm that
\begin{align*}
	&a_1\hat{\Vv}_1 + a_2\hat{\Vv}_2 + a_3\hat{\Vv}_3 \\
	&= -\frac{1}{\sqrt{3}} \times \frac{1}{\sqrt{3}} \vecthree{-1}{1}{1}   
		- \frac{5}{\sqrt{6}} \times \frac{1}{\sqrt{6}} \vecthree{1}{-1}{2} 
		+ \frac{7}{\sqrt{2}}\times\frac{1}{\sqrt{2}} \vecthree{1}{1}{0} \\
	&= 	-\frac{1}{3}\vecthree{-1}{1}{1} - \frac{5}{6}\vecthree{1}{-1}{2} + \frac{7}{2}\vecthree{1}{1}{0} \\
	&= \vecthree{3}{4}{-2} = \Vu
\end{align*}
}	
\end{question}

\begin{question}{6}{%
Given vectors
\[ \Vx_1 = \vectwo{-2}{1},\quad \Vx_2 = \vectwo{1}{1} \]
calculate the vector that is the projection of $\Vx_2$ onto $\Vx_1$.}{
\begin{align*}
	\proj{\Vx_2}{\Vx_1} &= \frac{\Vx_2\cdot\Vx_1	}{\Vx_1\cdot\Vx_1}\Vx_1 \\
	&= \frac{1\times-2 + 1\times1}{-2\times-2 + 1\times1}\Vx_1 \\
	&= \vectwo{2/5}{-1/5}
\end{align*}
}	
\end{question}

\begin{question}{7}{%
Using your answer to Question 6, find a vector nearest to $\Vx_2$ that is orthogonal to $\Vx_1$.}{
The vector $\Vx_2 - \proj{\Vx_2}{\Vx_1}$ is orthogonal to the vector $\Vx_1$.
\begin{align*}
	\Vx_2 - \proj{\Vx_2}{\Vx_1} &= \vectwo{1}{1} - \vectwo{2/5}{-1/5} \\
	&= \vectwo{3/5}{6/5}
\end{align*}
To verify, we compute the dot product between our new vector and $\Vx_1$.
\[ \vectwo{3/5}{6/5}\cdot\vectwo{-2}{1} = -6/5 + 6/5 = 0 \]
}
\end{question}

\begin{question}{8}{%
{\sc Challenge:} Look back at our proof of the theorem for decomposing a vector over an orthonormal basis using dot products. Can we use dot products to decompose over a basis where all vectors are orthogonal but not normalized?}{
We use the normality of the vectors after we show that
\[ \Vu\cdot\hat{\Vv}_i = a_i\hat{\Vv}_i\cdot\hat{\Vv}_i = a_i\norm{\hat{\Vv}_i}^2 \]
If the basis vectors were orthogonal but not normalized, we would instead have
\[ \Vu\cdot\Vv_i = a_i\Vv_i\cdot\Vv_i = a_i\norm{\Vv_i}^2 \]
Here we are unable to assume that $\norm{\Vv_i}^2 = 1$ since $\Vv_i$ is not a normal vector. But we can still solve for the coefficient $a_i$.
\[ a_i = (\Vu\cdot\Vv_i)/\norm{\Vv_i}^2 \]
The above formula uses only a dot product and a norm to compute each coefficient. These are far easier calculations than solving the linear system $\VV\Va = \Vu$ for non-orthogonal basis vectors. The power of orthonormal basis vectors comes from their orthogonality, not their normality.
}	
\end{question}



\end{document}
